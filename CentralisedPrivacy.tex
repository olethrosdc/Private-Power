\documentclass[a4paper,onecolumn]{article}

\usepackage{amsmath} 
\usepackage{amssymb} 
\usepackage{amsthm}
\usepackage{amsfonts}
\usepackage{graphicx}
\usepackage{subfigure}
\usepackage{subfigmat}
\usepackage{framed}
\usepackage[usenames,dvipsnames]{color}
\usepackage[textsize=small]{todonotes}
\usepackage{algorithm, algorithmic}
\numberwithin{algorithm}{section}
\usepackage{enumerate}
\usepackage{url}
%\usepackage{xcolor}
\usepackage[margin=3.5cm]{geometry}

\theoremstyle{plain}
\newtheorem{lem}{Lemma}
\newtheorem{thm}{Theorem}
\newtheorem{prop}{Proposition}
\theoremstyle{definition}
\newtheorem{defi}{Definition}
\newtheorem{problem}{Problem}
\newtheorem{assume}{Assumption}
\newtheorem{corollary}{Corollary}
\newtheorem{claim}{Claim}
\newtheorem{fact}{Fact}
\newtheorem{remark}{Remark}
\theoremstyle{example}
\newtheorem{example}{Example}
\newtheorem{obs}{Observation}
\newcommand{\loss} {\ell}

\newcommand {\defn} {\triangleq}
\newcommand{\Z}{\makebox[0.06cm][l]{\sf Z}{\sf Z}}
\newcommand{\Prob}{\makebox[0.06cm][l]{\sf P}{\sf P}}
\newcommand{\CS}{\mathcal{S}}
\newcommand{\CZ}{\mathcal{Z}}
\newcommand{\R}{\mathbb{R}}
\newcommand{\N}{\mathbb{N}}
\newcommand{\B}{\mathcal{B}}
\newcommand{\gam}{\gamma}
\newcommand{\bb}{\beta}
\newcommand{\oo}{\omega}
\newcommand{\lam}{\lambda}
\newcommand{\al}{\alpha}
\newcommand{\eps}{\epsilon}
\newcommand{\del}{\delta}
\newcommand{\sig}{\sigma}
\newcommand{\sgn}{\text{sgn}}
\newcommand{\m}{\mathcal}
\newcommand{\mb}{\mathbf}
\newcommand{\demand}{x}
\newcommand{\Demand}{\mathcal{X}}
\newcommand{\fdemand}{\beta}
\newcommand{\price}{p}
\newcommand{\cost}{J}
\newcommand{\nmech}{M}
\newcommand{\dpmech}{\nmech_{\eps,\delta}}
\newcommand{\neigh}{\mathcal{N}}
\newcommand{\sense}{\Delta}
\newcommand{\lmech}{\nmech^{L}_\eps}
\newcommand{\emech}{\nmech^{E}_\eps}
\newcommand{\pmech}{\nmech^{P}_\eps}
\newcommand{\baseline}{b}
\newcommand{\abs}[1] {\left|#1\right|}
\newcommand{\cset}[2] {\left\{#1 ~\middle|~ #2\right\}}
\newcommand{\maryam}[1]{\todo[inline,color=blue!20]{MK:\ #1}}
\newcommand{\christos}[1]{\todo[inline,color=red!20]{CD:\ #1}}
\newcommand{\maryamm}[1]{\todo[color=blue!20]{MK:\ #1}}
\newcommand{\christosm}[1]{\todo[color=red!20]{CD:\ #1}}

\newcommand{\Laplace}{\textrm{Laplace}}


\begin{document} 
\title{Notes: Pricey of Privacy in Real-time Electricity Markets}

\author{Christos and Maryam}
\maketitle

\section{Introduction.}

We consider the problem of privacy for households participating in
smart grid system. From the point of view of the households, there are
two distinct problems: firstly, how much privacy do they lose in
general by participating in an online price mechanism? Secondly, given
that they participate in the mechanism, how much privacy can they
expect with respect to their household occupancy?


\subsection{The price setting mechanism.}
In Cournot-type models, price depends on the total demand. This
dependence can be any increasing function. The consumers submit their
demand schedule, and the price structure is published. The question is
whether a given can learn anything about others' consumptions or
hidden variables (occupancy for example) based on observing the
price. 

Let the central utility have a quadratic loss function based on total
demand $\demand_t \defn \sum_i \demand_t^i$ of users
$i = 1, \ldots, n$, at time $t$.
\begin{equation}
  \label{eq:loss}
  \loss_t(x) = \cost(\demand_t + \fdemand_t)^2/2,
\end{equation}
where $\fdemand_t$ is a fixed demand term, that must be met independently \christos{What does this fixed demand represent?}
Then, for a large number of players, the price 
\begin{equation}
  \price_t = \cost(\demand_t + \fdemand_t),
  \label{eq:price}
\end{equation}
ensures that at the Nash equilibrium minimises $\loss_t$, and that the individual cost is
\begin{equation}
  c_t^i = \price_t x_t^i.
  \label{eq:cost}
\end{equation}

\subsection{Consumption privacy.}

Consider a mechanism like the above that, given the demanded consumption
$\demand_t^i$ of households $i$ at times $t$, decides upon a price
level $\price_{t}$ for that time period, which depends on
$\demand_{t}$. As the mechanism's price level is public knowledge, it
can leak private information about demand (or consumption). For that
reason, we shall consider price-setting algorithms that have privacy
guarantees.

If the mechanism for determining the price level is
$\epsilon$-differentially private for period $t$, the we must have
\[
\left|
  \ln \frac{\Pr(\price_t \mid \fdemand_t)}{\Pr(\price_t \mid x'_t)}
\right|
 \leq \epsilon
\]
where $\demand_t = (\demand^1_t, \ldots, \demand^n_t)$ is a vector of
bids and/or consumptions of individual households for the $t$-th time
period and $x'_t$ is a vector where the data of one individual is
added or missing.  However this only protects us against one time
period: for $T$ periods, this only guarantees $T \epsilon$
differential privacy.




\bibliographystyle{plain}
\bibliography{references}



\end{document}

%%% Local Variables:
%%% mode: latex
%%% TeX-master: t
%%% End:
