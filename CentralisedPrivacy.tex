\documentclass[a4paper,onecolumn]{article}

\usepackage{amsmath} 
\usepackage{amssymb} 
\usepackage{amsthm}
\usepackage{amsfonts}
\usepackage{graphicx}
\usepackage{subfigure}
\usepackage{subfigmat}
\usepackage{framed}
\usepackage[usenames,dvipsnames]{color}
\usepackage[textsize=small]{todonotes}
\usepackage{algorithm, algorithmic}
\numberwithin{algorithm}{section}
\usepackage{enumerate}
\usepackage{url}
%\usepackage{xcolor}
\usepackage[margin=3.5cm]{geometry}

\theoremstyle{plain}
\newtheorem{lem}{Lemma}
\newtheorem{thm}{Theorem}
\newtheorem{prop}{Proposition}
\theoremstyle{definition}
\newtheorem{defi}{Definition}
\newtheorem{problem}{Problem}
\newtheorem{assume}{Assumption}
\newtheorem{corollary}{Corollary}
\newtheorem{claim}{Claim}
\newtheorem{fact}{Fact}
\newtheorem{remark}{Remark}
\theoremstyle{example}
\newtheorem{example}{Example}
\newtheorem{obs}{Observation}


\newcommand{\Z}{\makebox[0.06cm][l]{\sf Z}{\sf Z}}
\newcommand{\Prob}{\makebox[0.06cm][l]{\sf P}{\sf P}}
\newcommand{\CS}{\mathcal{S}}
\newcommand{\CZ}{\mathcal{Z}}
\newcommand{\R}{\mathbb{R}}
\newcommand{\N}{\mathbb{N}}
\newcommand{\B}{\mathcal{B}}
\newcommand{\gam}{\gamma}
\newcommand{\bb}{\beta}
\newcommand{\oo}{\omega}
\newcommand{\lam}{\lambda}
\newcommand{\al}{\alpha}
\newcommand{\eps}{\epsilon}
\newcommand{\del}{\delta}
\newcommand{\sig}{\sigma}
\newcommand{\sgn}{\text{sgn}}
\newcommand{\m}{\mathcal}
\newcommand{\mb}{\mathbf}
\newcommand{\demand}{x}
\newcommand{\Demand}{\mathcal{X}}
\newcommand{\price}{p}
\newcommand{\cost}{J}
\newcommand{\nmech}{M}
\newcommand{\dpmech}{\nmech_{\eps,\delta}}
\newcommand{\neigh}{\mathcal{N}}
\newcommand{\sense}{\Delta}
\newcommand{\lmech}{\nmech^{L}_\eps}
\newcommand{\emech}{\nmech^{E}_\eps}
\newcommand{\pmech}{\nmech^{P}_\eps}
\newcommand{\baseline}{b}
\newcommand{\abs}[1] {\left|#1\right|}
\newcommand{\cset}[2] {\left\{#1 ~\middle|~ #2\right\}}
\newcommand{\maryam}[1]{\todo[inline,color=blue!20]{MK:\ #1}}
\newcommand{\christos}[1]{\todo[inline,color=red!20]{CD:\ #1}}
\newcommand{\maryamm}[1]{\todo[color=blue!20]{MK:\ #1}}
\newcommand{\christosm}[1]{\todo[color=red!20]{CD:\ #1}}

\newcommand{\Laplace}{\textrm{Laplace}}


\begin{document} 
\title{Notes: Pricey of Privacy in Real-time Electricity Markets}

\author{Christos and Maryam}
\maketitle

\section{Introduction.}

We consider the problem of privacy for households participating in
smart grid system. From the point of view of the households, there are
two distinct problems: firstly, how much privacy do they lose in
general by participating in an online price mechanism? Secondly, given
that they participate in the mechanism, how much privacy can they
expect with respect to their household occupancy?

\section{Consumption privacy.}

Consider a mechanism that, given the demanded consumption
$\demand_t^i$ of households $i$ at times $t$, decides upon a price
level $\price_{t}$ for that time period, which depends on
$\demand_{t}$. As the mechanism's price level is public knowledge, it
can leak private information about demand (or consumption). For that
reason, we shall consider price-setting algorithms that have privacy
guarantees.

If the mechanism for determining the price level is
$\epsilon$-differentially private for period $t$, the we must have
\[
\left|
  \ln \frac{\Pr(\price_t \mid \demand_t)}{\Pr(\price_t \mid x'_t)}
\right|
 \leq \epsilon
\]
where $\demand_t = (\demand^1_t, \ldots, \demand^n_t)$ is a vector of
bids and/or consumptions of individual households for the $t$-th time
period and $x'_t$ is a vector where the data of one individual is
added or missing.  However this only protects us against one time
period: for $T$ periods, this only guarantees $T \epsilon$
differential privacy.


\subsection{The price setting mechanism.}
The price-setting mechanism needs to be differentially private. The simplest way is to use the exponential mechanism. The utility of the price-setter is to ensure the demand peak is low.. this can be achieved by having a utility function $\left(\baseline_t + \frac{1}{N}\sum_{i=1}^N \demand_{i,t}\right)^2$, where $\baseline_t$ is an inflexible power demand and $\demand_{i,t}$ is the power demand of customer $i$ at time $t$. 
In general, a price-setting mechanism $\nmech$ itself determines the price $\price_t$ at time $t$ in a manner dependent upon a baseline price $\baseline_t$ at time $t$ and $\demand_{i,t}$, the expected demand by use $i$ at time $t$.
\[
\price_t = \nmech\left(\baseline_t + \frac{1}{N}\sum_{i=1}^N \demand_{i,t}\right),
\]


The baseline price may in turn depend on the overall supply and demand
for the time of day, while the individual demands are estimated from
local consumer data.  The cost for the $i$-th consumer is going to be
\[
\cost_i(\demand_i) = \sum_{t=1}^T \price_t \demand_{i,t}
\]
If we wish the price to be differentially private with respect to an individual's demand (at any given time period) then there are two standard methods whereby we can achieve this. The first is through adding Laplace noise to the price determined by the non-private mechanism $\nmech$. The second is by formulating the problem as an optimisation problem, whose solution as $\eps \to \infty$ converges to the mechanism $\nmech$.


\bibliographystyle{plain}
\bibliography{references}



\end{document}

%%% Local Variables:
%%% mode: latex
%%% TeX-master: t
%%% End:
